This chapter discusses three main sets of simulation results. First, we
compare the key neutronics results between Serpent and Moltres for a static
model of the \gls{MSFR}, i.e. no salt flow, to assess the accuracy of the
six-group neutron diffusion model in Moltres on a fast-spectrum reactor. This
exercise also serves as a checking exercise for the group constant data
generated for Moltres from Serpent. Second, we present steady-state results
from Moltres with salt flow and verify it against published results, namely
from the Polimi/TUDelft and OpenFOAM models developed by Fiorina et al.
\cite{fiorina_modelling_2014} and Aufiero et al.
\cite{aufiero_development_2014}, respectively. Lastly, we present and compare
transient simulation results from Moltres with the aforementioned models.
Transient accident scenarios include an unprotected reactivity insertion, an
unprotected loss of flow, an unprotected loss of heat sink, a chilled inlet,
and a pump overspeed scenario.

\section{Static Model}

We calculated estimates of $k_{\text{eff}}$ values in Serpent and Moltres for
the \gls{MSFR} model with static salt (no flow), a uniform temperature
distribution of 973 K, and a fuel salt density of $4.1249 \times 10^3$ kg
m$^{-3}$. Moltres performs eigenvalue calculations for finding the 
$k_{\text{eff}}$ as it does not currently have the capability to calculate
the adjoint flux. Table \ref{table:keff} shows the $k_{\text{eff}}$ values
from Serpent, and from Moltres with and without \glspl{DNP}. We observe a
small 127.9 pcm discrepancy between the two codes, which we attribute to two
main factors: the neutron diffusion model, and the omission of the blanket
tank structural material. It is well-known that the neutron diffusion model is
not as accurate as the other S$_{\text{N}}$ or SP$_{\text{N}}$ deterministic
methods nor the Monte Carlo approach in Serpent. Regarding the omission of the
blanket tank material, we replaced the 2 cm-thick structural material with
blanket salt. This decision may be partly responsible for the higher
$k_{\text{eff}}$ value calculated by Moltres. Nevertheless, the discrepancy is
smaller than the 228.5 pcm and 256.7 pcm discrepancy reported by Cervi et al.
\cite{cervi_development_2019} for their six-group $SP_3$ and neutron diffusion
methods, respectively, in OpenFOAM. The neutron diffusion model in OpenFOAM is
the same approach Aufiero et al. \cite{aufiero_development_2014} used for
their transient analysis of the \gls{MSFR}, albeit with one neutron energy
group. 

\begin{table}[htb!]
	\centering
	\caption{$k_{\text{eff}}$ values from Serpent and Moltres.}
	\begin{tabular}{l S S}
		\toprule
		{Code} & {$k_{\text{eff}}$} & {Diff. wrt Serpent [pcm]}\\
		\midrule
		{Serpent} & 1.00667 \pm 0.000049 & 0\\
		{Moltres with \glspl{DNP}} & 1.007949 \pm 0.000001 & 127.9\\
		{Moltres without \glspl{DNP}} & 1.004937 \pm 0.000001 & {-}\\
		\bottomrule
	\end{tabular}
	\label{table:keff}
\end{table}
%
\begin{table}[htb!]
	\centering
	\caption{$\beta_{\text{eff}}$ values from Serpent and Moltres.}
	\begin{tabular}{l S S}
		\toprule
		{Code} & {$\beta_{\text{eff}}$ [pcm]} & {Diff. wrt Serpent [pcm]}\\
		\midrule
		{Serpent} & 303.2 \pm .8 & 0\\
		{Moltres} & 301.2 \pm .2 & 2.0\\
		\bottomrule
	\end{tabular}
	\label{table:betaeff}
\end{table}

The 

\section{Steady-State Behavior}

\subsection{Neutronics}

\subsection{Thermal-Hydraulics}

\section{Transient Behavior}

\subsection{Unprotected Reactivity Insertion}

\subsection{Unprotected Loss of Flow}

\subsection{Unprotected Loss of Heat Sink}

\subsection{Chilled Inlet}

\subsection{Pump Overspeed}
