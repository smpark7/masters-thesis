This chapter compares key neutronics results between Serpent and Moltres
for a static model of the \gls{MSFR}, i.e. no salt flow, and uniform
temperature distribution to assess the accuracy of the
six-group neutron diffusion model in Moltres on a fast-spectrum reactor. This
exercise also serves as a verification for the group constant data
generated for Moltres from Serpent. For clarity, here are the important
details of the static model.

\textbf{Details of the static \gls{MSFR} model on Moltres}
\begin{itemize}
    \item No salt flow (static salt)
    \item Uniform temperature of 973 K throughout the 2D core model
    \item Six neutron energy groups
    \item Eight delayed neutron precursor groups
    \item Vacuum boundary conditions on the outer boundaries of the 2D core
    mesh
\end{itemize}

\section{Effective Multiplication Factor and Delayed Neutron Fraction}

Moltres solves the six-group neutron diffusion equations as a
steady-state eigenvalue problem to find the $k_{\text{eff}}$ for the static
\gls{MSFR} model. Table
\ref{table:keff} shows the $k_{\text{eff}}$ values from Serpent 2 and Moltres
at 973 K and the corresponding salt density, and Table \ref{table:ktemp} shows
the $k_{\text{eff}}$ values for other temperatures at 100 K intervals. We
observe small discrepancies on the order of 100 pcm between the two codes
which we attribute to two main factors: the accuracy of the neutron diffusion
model, and the omission of the blanket tank structural material in Moltres.
The neutron diffusion model is not as accurate as the other S$_{\text{N}}$ or
SP$_{\text{N}}$ deterministic methods nor the Monte Carlo approach in Serpent.
Regarding the omission of the blanket tank material, we replaced the
2 cm-thick structural material with blanket salt. This replacement may be
partly responsible for the higher $k_{\text{eff}}$ value calculated by Moltres
as the macroscopic fission cross sections of the blanket salt are non-zero for
the higher neutron energy groups. Nevertheless, the discrepancy is smaller
than the 228.5 pcm and 256.7 pcm discrepancies reported by Cervi et al.
\cite{cervi_development_2019} for their six-group $SP_3$ and neutron
diffusion methods, respectively, in OpenFOAM. Aufiero et al.
\cite{aufiero_development_2014} used the same neutron diffusion model in
OpenFOAM for their transient analysis of the \gls{MSFR}, albeit with one
neutron energy group to reduce computational load.

\begin{table}[htb!]
    \small
	\centering
	\caption{$k_{\text{eff}}$ values from Serpent 2 and Moltres at 973 K.}
	\begin{tabular}{l S c}
		\toprule
		{Code} & {$k_{\text{eff}}$} \\
		\midrule
		{Serpent 2} & 1.00662 \pm 0.00005 \\
		{Moltres with \glspl{DNP}} & 1.0079400 \pm 0.0000010 \\
		{Moltres without \glspl{DNP}} & 1.0049197 \pm 0.0000010 \\
		\bottomrule
	\end{tabular}
	\label{table:keff}
\end{table}
%
\begin{table}[htb!]
    \small
	\centering
	\caption{$k_{\text{eff}}$ values from Serpent 2 and Moltres at various
	temperatures from 800 K to 1400 K.}
	\begin{tabular}{S S S S}
		\toprule
		{Temperature [K]} & {$k_{\text{eff}}$ $\pm$ $\sigma$
		(Serpent 2)} & {$k_{\text{eff}}$ (Moltres)} & {Difference wrt Serpent
		2 [pcm]}
		\\
		\midrule
		800  & 1.01996 \pm 0.00005 & 1.02117 & 121 \\
		900  & 1.01172 \pm 0.00005 & 1.01322 & 150 \\
		1000 & 1.00428 \pm 0.00005 & 1.00544 & 116 \\
		1100 & 0.99735 \pm 0.00005 & 0.99859 & 124 \\
		1200 & 0.99006 \pm 0.00005 & 0.99119 & 113 \\
		1300 & 0.98356 \pm 0.00005 & 0.98439 &  83 \\
		1400 & 0.97702 \pm 0.00005 & 0.97820 & 118 \\
		\bottomrule
	\end{tabular}
	\label{table:ktemp}
\end{table}

The absolute value of $k_{\text{eff}}$ impacts the final steady-state
temperature of the reactor. We can raise or lower the average core temperature
at steady state to meet the design specifications for the inlet and outlet
temperatures by adjusting the fissile inventory and $k_{\text{eff}}$. In
transient analysis, the delayed neutron fraction $\beta$ and reactivity
coefficients $\alpha$ are
important as they dictate the duration, shape, and magnitude of the reactor
transient response. The difference between $\beta$ and
$\beta_{\text{eff}}$ is that $\beta$ is the physical delayed neutron fraction
while $\beta_{\text{eff}}$ is the delayed neutron fraction weighted on the
adjoint neutron flux. We will compare the $\beta$ value from Moltres to the
$\beta_{\text{eff}}$ value from Serpent because Moltres currently lacks a
adjoint calculation capability. We calculated $\beta$ by
taking the relative difference between the $k_{\text{eff}}$ values with and
without \glspl{DNP} in Table \ref{table:keff}. The $\beta$ and
$\beta_{\text{eff}}$ values at 973 K, shown in Table \ref{table:betaeff}, are
in good agreement with a 4.43 pcm discrepancy.

\begin{table}[htb!]
	\centering
	\caption{$\beta_{\text{eff}}$ and $\beta$ values from Serpent 2 and
	Moltres, respectively, at 973 K.}
	\begin{tabular}{l S S}
		\toprule
		{Code} & {$\beta_{\text{eff}}$ [pcm]} & {Difference wrt Serpent [pcm]}
		\\
		\midrule
		{Serpent} & 304.08 \pm .81 & {-}\\
		{Moltres} & 299.65 \pm .20 & 4.43\\
		\bottomrule
	\end{tabular}
	\label{table:betaeff}
\end{table}

\section{Reactivity Feedback Coefficients}

The temperature reactivity feedback arises mainly from Doppler broadening of
resonance absorption peaks and temperature-induced density changes. Although
Doppler coefficients typically show logarithmic dependence to temperature, we
report them, along with the other coefficients, as linear gradient
values of the reactivity given the relatively linear trend within the relevant
temperature range (Figure \ref{fig:reactivity}). The temperature range
extends below the melting point of the fuel salt (873 K) to ensure that
the data covers the relevant range between 873 K and 900 K. Table
\ref{table:alpha} shows
the temperature coefficients as described prior. The total temperature
coefficients from Serpent and Moltres show excellent agreement with a
discrepancy of 0.019 pcm K$^{-1}$.

\begin{figure}[htb!]
    \centering
    \includegraphics[width=.8\textwidth]{reactivity}
    \caption{Reactivity values from Serpent and Moltres. The Doppler
    reactivity values were calculated at a fixed density of 4.1249 g
    cm$^{-3}$. The density reactivity values were calculated at a fixed
    temperature of 973 K.}
    \label{fig:reactivity}
\end{figure}
%
\begin{table}[htb!]
	\centering
	\caption{Doppler, density, and total temperature coefficients
	for the temperature range of 800 K to 1400 K.}
	\begin{tabular}{l S S S S}
		\toprule
		{Software} & {$\alpha_D$ (log) [pcm]} & {$\alpha_D$ (linear) [pcm
		K$^{-1}$]} & {$\alpha_\rho$ [pcm K$^{-1}$]} & {$\alpha_T$ [pcm
		K$^{-1}$]} \\
		\midrule
		{Serpent} & -4034 \pm 14 & -3.737 \pm 0.013 & -3.424 \pm 0.013 &
		-7.165 \pm 0.013 \\
		{Moltres} & {-} & {-} & {-} & -7.184\\
		\bottomrule
	\end{tabular}
	\label{table:alpha}
\end{table}

\section{Neutron Energy Spectrum}

\begin{figure}[htb!]
    \centering
    \includegraphics[width=.8\textwidth]{nt-spec}
    \caption{The fine-group and six-group neutron energy spectra from Serpent
    2 and Moltres normalized per unit lethargy.}
    \label{fig:ntspec}
\end{figure}

Moltres also closely replicated the six-group neutron spectrum from the
Serpent group constants. Figure \ref{fig:ntspec} compares the neutron energy
spectra from Serpent and Moltres in the central fuel salt region. The
six-group neutron spectra overlap exactly over each other. More generally, the
plot shows the distinctive fast spectrum observed in the \gls{MSFR} with dips
in the spectrum corresponding to elastic scattering resonances from lithium
and fluorine. From this plot, we observe that the discrepancies in
$k_{\text{eff}}$ arise mainly from discretizing neutron energy into groups
rather than the neutron diffusion model itself. We could obtain a more
accurate representation of the neutronics in the \gls{MSFR} by using more
neutron energy groups but this would adversely impact simulation times in the
subsequent multiphysics finite element analyses.

In summary, Moltres replicated most of the relevant neutronics parameters
accurately with the group constant data from Serpent. Moltres agrees with the
high fidelity simulation in Serpent 2 for the $\beta_{\text{eff}}$ and
temperature reactivity coefficients, which are important parameters for
modeling transient reactor behavior. The $k_{\text{eff}}$ values have
discrepancies on the order of 100 pcm and they are relatively small compared
to the $k_{\text{eff}}$ values from the neutron diffusion and SP3 models in
OpenFOAM \cite{cervi_development_2019}.
