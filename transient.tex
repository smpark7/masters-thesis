This chapter discusses the transient multiphysics simulation results of the
\gls{MSFR} from Moltres for four accident scenarios. These scenarios include
unprotected instances of reactivity insertion, loss of flow, pump overspeed,
and loss of heat sink. The term ``unprotected'' means that there is no
external intervention in these scenarios. As such, these simulations
give an insight on the \gls{MSFR}'s passive safety capabilities in the absence
of any active safety system. We used the steady-state configuration presented
in the previous chapter as the initial conditions for the transient
simulations discussed in this chapter. In essence, we imported all
steady-state spatial values for neutron flux, delayed neutron precursor
concentration, temperature, velocity, and pressure as the initial state of the
transient scenarios.

As noted by Fiorina et al. \cite{fiorina_modelling_2014}, explicit decay heat
modeling has a negligible effect on reactivity-, pump-, and enhanced
cooling-initiated transients. Furthermore, only their Polimi model had decay
heat modeling capabilities. Therefore, they presented results from their
Polimi and TUDelft model for the four accident scenarios (listed above)
without decay heat modeling. They only enabled decay heat modeling for the
loss of heat sink transient scenario only. For a fair comparison, we also ran
all transient scenarios without the decay heat model with the exception of the
loss of heat sink scenario in which we ran the simulations with and without
the decay heat model.

We detailed the setup for each transient simulation in their respective
sections.

\section{Unprotected Reactivity Insertion}

\glspl{RIA} are a type of nuclear accident caused by unintended positive
reactivity insertions. The excess reactivity causes the power output and
temperatures in nuclear reactors to rise to potentially dangerous levels. In
\glspl{MSR}, a positive reactivity insertion could occur when the online
refueling system injects excess fissile material into the core. Excessively
high neutron fluences and temperatures negatively impact reactor structural
integrity and increase the risk of containment breach.

We modeled two unprotected reactivity insertion scenarios in Moltres by
swapping out the original set of group constant data with two new, separate
sets of data from Serpent corresponding to 50 pcm and 200 pcm reactivity
insertions, respectively. We increased the reactivity of the Serpent
\gls{MSFR} models by increasing the $^{233}$U-to-$^{232}$Th ratio in the fuel
salt.


\section{Unprotected Loss of Flow}

A loss of forced flow accident can occur in the event of a station
blackout; the pumps would cease operating due to the loss of AC electrical
power. Natural circulation resulting from temperature-dependent density
changes becomes the dominant driving force for salt flow in the primary loop.
While Fiorina et al. \cite{fiorina_modelling_2014} and Aufiero et al.
\cite{aufiero_development_2014} applied the Boussinesq approximation for
bouyancy-driven flow in their models, it is not directly applicable to the
\gls{MSFR} model in this thesis because we partitioned the primary loop into
two separate geometries and used Dirichlet boundary conditions to drive flow.
Fiorina et al.'s Polimi and TUDelft models featured complete exponential
coast-downs of the pumps with a time constant of 5 s. The resulting flow rate
from natural circulation was approximately 18 times smaller than the initial
flow rate. Thus, for the \gls{MSFR} model in Moltres, we imposed a similar
exponential decay term with the same time constant on the inflow Dirichlet
boundary condition to reduce the flow rate to the same final flow rate
observed by Fiorina et al. Figure * shows the change in flow rate over the
transient simulation.

We also expect the reduced flow rate to decrease the heat transfer rate
between the primary and secondary loop through the heat exchanger as the heat
transfer coefficient is dependent on the flow rate. Given the limited
information available on the heat exchanger specifications, we assumed that
the heat transfer coefficient is directly proportional to the flow rate, and
the overall heat transfer rate retains its linear dependence to the
temperature difference between the primary and secondary loop at the location
of the heat exchanger as shown previously in \ref{eq:hx}.



\section{Unprotected Pump Overspeed}



\section{Unprotected Loss of Heat Sink}

\subsection{Without Decay Heat Model}

\subsection{With Decay Heat Model}
