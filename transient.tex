This chapter discusses Moltres transient multiphysics simulations of the
\gls{MSFR} in four accident scenarios. These scenarios, adapted from the
\gls{MSFR} transient analyses with the Polimi and TUDelft models
\cite{fiorina_modelling_2014}, include unprotected reactivity
insertion, loss of heat sink, loss of flow, and pump overspeed accidents. The
term ``unprotected'' signify accident scenarios without treactor SCRAM. As
such, these simulations provide an insight on the
\gls{MSFR}'s passive safety capabilities in the absence
of active safety systems. This work used the steady-state configuration
presented in the previous chapter as the initial conditions for the transient
simulations discussed in this chapter. Specifically, all steady-state spatial
values for neutron flux, delayed neutron precursor concentration, temperature,
velocity, and pressure were imported as the initial state of the transient
scenarios.

As noted by Fiorina et al. \cite{fiorina_modelling_2014}, explicit decay heat
modeling has a negligible effect in reactivity-, and pump-initiated
transients. Furthermore, only their Polimi model had this capability.
Therefore, they featured decay heat modeling for the loss of heat sink
accident scenario only. The current work also ran all transient simulations
without the decay heat model for a fair comparison. The only exception is the
loss of heat sink scenario in which two separate simulations with and without
the decay heat model were run. This work imposed simplifying assumptions in
our transient models to match the implementations in
\cite{fiorina_modelling_2014} as closely as possible, within Moltres'
capabilities. The details of the setup for each
transient simulation appear in the following sections.

\section{Unprotected Reactivity Insertion}
In reactivity insertion accidents, excess reactivity would drive an increase
in flux, power, and temperature. In
\glspl{MSR}, a positive reactivity insertion could occur if the online
refueling system injected excess fissile material into the core. Excessively
high neutron fluences and temperatures could negatively impact reactor
structural integrity and increase the risk of containment breach.

\begin{figure}[htbp!]
    \centering
    \includegraphics[width=.7\textwidth]{insertion}
    \caption{Step-wise 50 pcm and 200 pcm reactivity insertions used to
    initiate the accident transients.}
    \label{fig:insertion}
\end{figure}

\begin{figure}[htbp!]
    \centering
    \includegraphics[width=.7\textwidth]{50pcm-jump}
    \caption{Power output during the prompt response following a 50 pcm
    step-wise unprotected reactivity insertion in the Moltres, Polimi, and
    TUDelft models \cite{fiorina_modelling_2014}.}
    \label{fig:50pcmjump}
\end{figure}

This work modeled two unprotected step-wise reactivity insertion scenarios in
Moltres by swapping out the original set of group constant data with two new,
separate sets of data from Serpent corresponding to 50 pcm and 200 pcm
reactivity insertions, respectively. The reactivity of the Serpent \gls{MSFR}
models was increased by increasing the $^{233}$U-to-$^{232}$Th ratio in
the fuel salt. Figure \ref{fig:insertion} shows the step-wise reactivity
insertions used to model the accident scenarios.

The neutronic and thermal-hydraulic behaviors of the reactor core are the
focus of this transient study. Thus, this work assumes that the heat exchanger
and the associated power generation equipment (generator turbines, heat sinks,
etc.) can withstand all variations in the power output during the transients.

Figure \ref{fig:50pcmjump} shows the rise in power output during
the initial prompt response. The prompt
response raises the power output to 4 GW by $t=0.001$ s.
Figures \ref{fig:50pcmheat} and \ref{fig:50pcmtemp} show the power output and
average core temperature increase beyond $t=0.001$ s following the 50 pcm
step-wise reactivity insertion in the Moltres, Polimi, and TUDelft models.
Power continues to rise at a slower rate up to 4.63 GW at around $t=0.005$ s,
at which point the negative reactivity from the Doppler effect and salt
expansion becomes greater
than the initial +50 pcm insertion. Power continues to fall as the average
core temperature rises. A slight change in slope occurs at $t=0.3$ s. The
elapsed time approximately corresponds to the average half-life of the two
shortest-lived delayed neutron precursor (DNP) groups ($t_{1/2}=0.195$ s and
$0.424$ s); the decay of the surplus precursors produced in the initial phase
negates a fraction of the negative reactivity from the elevated core
temperature. By $t=3$ s, most of the heated salt and \glspl{DNP} from the
initial phase will have circulated around the primary loop and returned to the
core. The heated salt causes a small, noticeable dip in power before the power
output stabilizes.

The average core temperature rises steadily from the start of the transient
until $t=3$ s when the heated salt from the initial phase returns to the core.
This event is characterized by the small peak in the average core temperature
at $t=3$ s. The subsequent drop in power output halts the temperature
increase and the core tends to a new equilibrium average temperature
approximately 7.5 K higher than the initial average temperature.

\begin{figure}[htbp!]
    \centering
    \includegraphics[width=.85\textwidth]{50pcm-heat}
    \caption{Power output following a 50 pcm step-wise unprotected reactivity
    insertion in the Moltres, Polimi, and
    TUDelft models \cite{fiorina_modelling_2014}.}
    \label{fig:50pcmheat}
%
    \centering
    \includegraphics[width=.85\textwidth]{50pcm-temp}
    \caption{Average core temperature increase following a 50 pcm step-wise
    unprotected reactivity insertion in the Moltres, Polimi, and
    TUDelft models \cite{fiorina_modelling_2014}.}
    \label{fig:50pcmtemp}
\end{figure}

The results from Moltres show good agreement with the results from the Polimi
and TUDelft models; Moltres reproduced all of the individual features in both
plots. The magnitude of the reactor response is the most significant
difference. Moltres predicts a smaller peak in the power output and a smaller
overall increase in the average core temperature mainly due to the
more negative temperature reactivity coefficient in Moltres than in the Polimi
and TUDelft models. The temperature reactivity coefficient $\alpha_T$ in
Moltres is $-7.184$ pcm K$^{-1}$ (Table \ref{table:alpha}), as opposed to
approximately $-6.5$ pcm K$^{-1}$ within the relevant temperature range in the
Polimi and TUDelft models. Therefore, the results show a smaller temperature
increase in the Moltres model for the same reactivity insertion. Multiplying
the average
core temperature increase at $t=10$ s with $\alpha_T$ gives us $-7.184$ pcm
K$^{-1} \times 7.46$ K $= -53.6$ pcm, which is approximately equal to the 50
pcm reactivity insertion.

\begin{figure}[htbp!]
    \centering
    \includegraphics[width=.85\textwidth]{200pcm-heat}
    \caption{Power output following a 200 pcm step-wise unprotected reactivity
    insertion in the Moltres, Polimi, and
    TUDelft models \cite{fiorina_modelling_2014}.}
    \label{fig:200pcmheat}
%
    \centering
    \includegraphics[width=.85\textwidth]{200pcm-temp}
    \caption{Average core temperature increase following a 200 pcm step-wise
    unprotected reactivity insertion in the Moltres, Polimi, and
    TUDelft models \cite{fiorina_modelling_2014}.}
    \label{fig:200pcmtemp}
\end{figure}

The results for the 200 pcm reactivity insertion scenario show similar trends
to the 50 pcm case. The greater reactivity insertion elicits a stronger
prompt response in the power output which peaks at 92.1 GW. The average core
temperature increases much more rapidly and subsequently triggers a sharper
drop in power output. This more clearly distinguishes the rate of
core temperature increase before and after $t=0.01$ s. The difference in
$\alpha_T$ causes greater deviations in the results between Moltres and the
other models. Overall,
Moltres' results show good agreement with the Polimi and TUDelft results.
The differences arise mainly due to the differences in the temperature
reactivity coefficients.

\clearpage

\section{Unprotected Loss of Heat Sink}

An unprotected loss of heat sink accident can occur when the pumps in the
intermediate loop fail and reactor is not SCRAMed. The heat exchangers would
then lose most of their
cooling capabilities. This work followed Fiorina et al.'s approach in assuming
that the cooling from the heat exchangers decreases exponentially with a time
constant of 1 s and all other parameters held constant
\cite{fiorina_modelling_2014}. As mentioned in the Chapter \ref{chap:ss}, we
will present two sets of results for this transient: 1) without decay heat
modeling, and 2) with decay heat modeling.

\subsection{Without Decay Heat} \label{sec:wodecayheat}

Figures \ref{fig:lohsheat} and \ref{fig:lohstemp} show the power output and
average core temperature increase during the unprotected loss of heat sink
transient in the Moltres, Polimi, and TUDelft models without decay heat
modeling. The power output and average core temperature show little change in
the first two seconds as it takes approximately that amount of time for the
partially cooled salt to migrate to the center of the core. At $t=2$ s, we
observe a sharp spike in average core temperature and a corresponding drop
in power output. The presence of delayed neutron precursors (DNPs) from the
steady-state operating conditions momentarily halt the increase in temperature
at around $t=5$ s. The average core temperature continues to rise while the
power output falls through the rest of the transient.

The results from Moltres show good agreement with the results from the Polimi
and TUDelft. Moltres reproduced all of the trends in the Polimi
and TUDelft models. The temporary halt in the temperature increase occurs at
a lower average core temperature for Moltres than the other two models. This
is likely due to the difference in the temperature reactivity coefficient
discussed in the reactivity insertion results; a smaller increase in
the average core temperature produces the same decrease in power output.

\begin{figure}[htbp!]
    \centering
    \includegraphics[width=.85\textwidth]{lohs-heat}
    \caption{Power output during
    an unprotected loss of heat sink transient in the Moltres, Polimi, and
    TUDelft models \cite{fiorina_modelling_2014} without decay heat.}
    \label{fig:lohsheat}
    \includegraphics[width=.85\textwidth]{lohs-temp}
    \caption{Average core temperature increase during
    an unprotected loss of heat sink transient in the Moltres, Polimi, and
    TUDelft models \cite{fiorina_modelling_2014} without decay heat.}
    \label{fig:lohstemp}
\end{figure}

\clearpage

\subsection{With Decay Heat}

Decay heat from fission products poses a great safety risk in an unprotected
loss of
heat sink accident. Section \ref{sec:wodecayheat} showed that prompt fission
power output quickly falls as core temperatures rise. However, decay power
output is independent of the instantaneous neutron flux. Figure
\ref{fig:moltresdecaypower} shows that the decay power output remains
relatively high during a short-term transient. Decay heat becomes the dominant
heat source from $t=34$ s and falls at a much slower rate than prompt heat.
Figure \ref{fig:moltresdecaytemp} highlights the greater core temperature
increase arising from decay heat as compared with
the results without decay heat. By $t=120$ s, the model with decay heat
records an average core temperature increase that is 45 K higher than the
model without decay heat. The absolute core temperature reaches approximately
1220 K and would continue to rise further. This places undue thermal stress
and accelerates salt-induced corrosion in the Hastelloy structural material.
In the
absence of an auxiliary heat removal system in the primary loop, reactor
operators would have to rely on the freeze plug to drain the core into a drain
tank with emergency cooling systems to keep the salt cool.

Figure \ref{fig:polimidecaytemp} shows the loop-averaged temperature
increase in the Moltres and Polimi models \cite{fiorina_modelling_2014}. The
TUDelft model does not have a decay heat modeling capability. The Moltres
model predicts the same increasing trend in the temperature. The loop-averaged
temperature rises significantly at the start of the transient and continues to
rise at a decreasing rate. The loop-averaged temperature increase in the
Moltres model at $t=120$ s is approximately 17 K lower than that in the Polimi
model. It is difficult to ascertain the exact cause for this difference
without the power profile from the Polimi model with decay heat to compare
with. However, if the decay power output are similar, the stronger negative
temperature reactivity coefficient would cause the prompt power output in the
Moltres model to fall faster than the Polimi model. Consequently, the
loop-averaged temperature would be lower as shown in the figure. Overall, the
results for the loss of heat sink transient agree with the Polimi and TUDelft
model results in both cases, with and without decay heat modeling.

\begin{figure}[htbp!]
    \centering
    \includegraphics[width=.85\textwidth]{moltres-decay-power}
    \caption{Power output during
    an unprotected loss of heat sink transient in the Moltres model with and
    without decay heat.}
    \label{fig:moltresdecaypower}
    \includegraphics[width=.85\textwidth]{moltres-decay-temp}
    \caption{Average core temperature increase during
    an unprotected loss of heat sink transient in the Moltres model with and
    without decay heat.}
    \label{fig:moltresdecaytemp}
\end{figure}

\clearpage

\begin{figure}[htbp!]
    \centering
    \includegraphics[width=.85\textwidth]{decay-temp}
    \caption{Loop-averaged temperature increase during
    an unprotected loss of heat sink transient in the Moltres and Polimi
    models \cite{fiorina_modelling_2014} with decay heat.}
    \label{fig:polimidecaytemp}
\end{figure}

\clearpage

\section{Unprotected Pump Overspeed}

Pump overspeed refers to a sustained
increase in pump speed in the primary coolant loop. The increased flow rate
$\dot{m}$ impacts reactor performance in several ways.
It affects the neutronics by reducing the in-core $\beta$ as more of the
shorter-lived precursors will tend to flow out of the core before decaying.
This net loss of neutrons reduces the reactivity in the core, thereby causing
core temperatures to fall to counteract this change through temperature
reactivity feedback. The increased $\dot{V}$ also enhances the heat transfer
coefficient on the primary loop side of the heat exchanger and enables the
reactor to operate at a higher power output. At the same time, the improved
mixing flattens the temperature distribution in the core.

This work followed Fiorina et al.'s implementation
\cite{fiorina_modelling_2014} by
ramping up the inlet velocity, $u$, by 50\% from the nominal value, $u_0$,
according to the following formula:
%
\begin{align}
    u(t) &= u_0 [1 + 0.5 (1 - e^{-t / \tau})] \label{eq:overspeed}
    \intertext{where}
    \tau &= 5 \text{ s.} \nonumber
\end{align}
%
For this transient, this work assumed that $\mu_t$ was directly proportional
to $v$ because the buoyancy effects are assumed to be negligible with forced
flow and the recirculation zones 
persist throughout the entire duration.

Setting the exact dependence between the heat transfer coefficient $h$ and
$\dot{V}$ was problematic because
the pointwise heat exchanger implementation in Moltres performs differently
compared with the heat exchangers of finite volume in the Polimi and TUDelft
models \cite{fiorina_modelling_2014}. In a heat exchanger of finite volume,
most of the cooling occurs in the first half of the heat exchanger where the
temperature differential between the primary and intermediate loops is the
largest. In the Polimi and TUDelft models, the overall $h$ is a
harmonic mean of the $h_i$ on each side of the heat exchanger, given as:
%
\begin{align}
    h &= \frac{2}{\frac{1}{h_1} + \frac{1}{h_2}}, \\
    \intertext{where}
    h &= \text{overall heat transfer coefficient
    [W$\cdot$K$^{-1}$],} \nonumber \\
    h_1 &= \text{heat transfer coefficient on the primary loop side
    [W$\cdot$K$^{-1}$],} \nonumber \\
    h_2 &= \text{heat transfer coefficient on the intermediate loop side
    [W$\cdot$K$^{-1}$].} \nonumber
\end{align}
%
In addition to this, Fiorina et al. applied the Dittus-Boelter correlation
\cite{dittus_heat_1930} for the relationship between $h_1$ and
$\dot{V}$. The Dittus-Boelter correlation for fluids being cooled is:
%
\begin{align}
    Nu &= 0.023 Re^{0.8} Pr^{0.3}, \label{eq:db} \\
    \intertext{where}
    Nu &= \text{Nusselt number,} \nonumber \\
    &\text{\ \ \ \ \ the ratio of convective to conductive heat
    transfer at a boundary in a fluid,} \nonumber \\
    Re &= \text{Reynolds number,} \nonumber \\
    &\text{\ \ \ \ \ the ratio of inertial forces to viscous
    forces within a fluid,} \nonumber \\
    Pr &= \text{Prandtl number,} \nonumber \\
    &\text{\ \ \ \ \ the ratio of momentum diffusivity to thermal
    diffusivity.} \nonumber
\end{align}
%
The only direct relation to $\dot{V}$ in the Dittus-Boelter correlation is
through the Reynolds number, $Re$, which is directly proportional to
flow velocity $v$. This gives the following relation between $h$ and $v$:
%
\begin{align}
    h &\propto v^{0.8} \label{eq:hv} \\
    h &= \text{heat transfer coefficient [W$\cdot$K$^{-1}$],} \nonumber \\
    v &= \text{flow velocity [m$\cdot$s$^{-1}$].} \nonumber
\end{align}
%
However, this relation provided very different results in the unprotected pump
overspeed and loss of flow transients compared with the Polimi and TUDelft
models. This approach underpredicted the equilibrium power output in the
unprotected pump overspeed transient and overpredicted the same parameter in
the unprotected loss of flow transient. Upon further investigation, I
found that raising the power of $v$ from 0.8 to 1.1 brought the
average core temperatures closer to the results from the other models in both
transients. Therefore, this work adopted the raised power in this thesis.

Figures \ref{fig:poheat} and
\ref{fig:potemp} show the power output and average core temperature increase
during the unprotected pump overspeed transient in the Moltres, Polimi, and
TUDelft models. Figure \ref{fig:poshort} shows the same results for the first
20 seconds of the transient. At the start of the transient, the rising flow
rate cools the core and causes power output to rise sharply. Although the
average core temperature has a strictly decreasing trend, the temperature at
the center of the core briefly rises due to the sharp increase in power
output. Since this is the region where most of the fissions take place, the
Doppler effect and salt expansion causes the power output to stall and dip
briefly before rising again at $t=2.5$ s. The reactor tends to a new
equilibrium power output and average core temperature. The temperature
distribution in the core is more evenly distributed because the turbulent
thermal conductivity $k_t$ is directly proportional to $mu_t$.

\begin{figure}[htbp!]
    \centering
    \includegraphics[width=.85\textwidth]{po-heat}
    \caption{Power output during
    an unprotected pump overspeed transient in the Moltres, Polimi, and
    TUDelft models \cite{fiorina_modelling_2014}.}
    \label{fig:poheat}
    \includegraphics[width=.85\textwidth]{po-temp}
    \caption{Average core temperature increase during
    an unprotected pump overspeed transient in the Moltres, Polimi, and
    TUDelft models \cite{fiorina_modelling_2014}.}
    \label{fig:potemp}
\end{figure}

\begin{figure}[htbp!]
    \centering
    \begin{subfigure}[t]{.485\textwidth}
        \centering
        \includegraphics[width=\textwidth]{po-heat-short}
    \end{subfigure}
    \hfill
    \begin{subfigure}[t]{.485\textwidth}
        \centering
        \includegraphics[width=\textwidth]{po-temp-short}
    \end{subfigure}
    \caption{The first 20 s of the power output and average core temperature
    increase during an unprotected pump overspeed transient.}
    \label{fig:poshort}
\end{figure}

In both sets of results, Moltres reproduced all of the transient features
found in the Polimi and TUDelft models. The average core temperature profile
falls between the Polimi and TUDelft results, while the power output is
approximately 0.1 GW higher because the Moltres \gls{MSFR} model has a
greater $\alpha_T$ than the other two models. 

\section{Unprotected Loss of Flow}

A loss of forced flow transient can occur in the event of a station blackout
without reactor SCRAM; the pumps would cease operating due to the loss of AC
electrical
power. Natural circulation resulting from temperature-dependent density
changes becomes the dominant driving force for salt flow in the primary loop.
Fiorina et al. \cite{fiorina_modelling_2014} applied the Boussinesq
approximation for buoyancy-driven flow in the Polimi and TUDelft models, but
this approach was not possible in Moltres because the primary loop is
partitioned into two separate geometries and used Dirichlet boundary
conditions at the inlet to drive flow. The Polimi and TUDelft
models featured complete exponential coast-downs of the pumps with a time
constant of 5 s. The final flow rate $\dot{V}_f$ from natural circulation
was approximately 18 times smaller than the initial $\dot{V}_0$. Figure
\ref{fig:flowrate} shows that the actual $\dot{V}$ decreased with a time
constant of 8 s. Thus, for the \gls{MSFR} model in Moltres, this work imposed
a similar exponential decay term with a time constant of 8 s on the inflow
Dirichlet boundary condition:
%
\begin{align}
    \dot{V} &= \dot{V}_f + (\dot{V}_0-\dot{V}_f) e^{-t/8},
    \label{eq:flowrate} \\
    \intertext{where}
    \dot{V}_f &= \text{final volumetric flow rate} = 0.25862
    \text{ m$^3\cdot$s$^{-1}$,} \nonumber \\
    \dot{V}_0 &= \text{initial volumetric flow rate} = 4.5
    \text{ m$^3\cdot$s$^{-1}$.} \nonumber
\end{align}

\begin{figure}[htbp!]
    \centering
    \includegraphics[width=.7\textwidth]{lof-flow-rate}
    \caption{The change in flow rate in the Polimi and TUDelft models and the
    imposed flow rate in Moltres.}
    \label{fig:flowrate}
\end{figure}
%
The reduced $\dot{V}$ also decreases the heat transfer rate between the
primary and intermediate loop through the heat exchanger as the heat transfer
coefficient $h$ is dependent on the $\dot{V}$. For this loss of flow
transient, Fiorina et al. \cite{fiorina_modelling_2014} intended to focus
on the primary loop and assumed that only the pumps in the primary loop
failed.

An issue arose pertaining to the turbulent viscosity $\mu_t$ as a function of
$v$. When using a simple approximation of $\mu_t$ being directly proportional
to $v$, the results from Moltres differed significantly compared with the
Polimi and TUDelft models. The difference is due to buoyancy-driven flow
contributing to turbulence; the turbulent energy $k$ equation in COMSOL's
$k$-$\epsilon$ model has an explicit source term from buoyancy effects
\cite{comsol_ab_comsol_2018}. Another point to note is the
Reynolds number remains constant if $\mu$ and $v$ decrease in tandem. This
preserves the existence of the recirculation zone in the core and it is at
odds with the results from the Polimi and TUDelft models, which show that the
recirculation zones disappear during the loss of flow transient. The current
work circumvented this issue by letting fixed fractions of the initial
$\mu_{t,0}$ be conserved regardless of the final flow velocity, according to
the following equation:
%
\begin{align}
    \mu_t &= \mu_c + (\mu_{t,0} - \mu_c) e^{-t/8} \label{lofmu} \\
    \intertext{where}
    \mu_c &= \text{ conserved fraction of $\mu_{t,0}$ [Pa$\cdot$s].} \nonumber
\end{align}
%
This measure allowed for laminar flow to develop in the core and yielded
results showing better qualitative agreement with those from the Polimi and
TUDelft models. The subsequent paragraphs discuss these results.

Figures \ref{fig:lofheat} and \ref{fig:loftemp} show the power output and
average core temperature increase during the unprotected loss of flow
transient in the Moltres, Polimi, and TUDelft models without decay heat
modeling. The three sets of results from Moltres correspond to $\mu_c =
\frac{1}{4} \mu_{t,0}, \frac{1}{2} \mu_{t,0}, \text{and } \frac{3}{4}
\mu_{t,0}$.
Although Moltres shows the same decreasing trend in power output, it failed to
capture the exact individual features in the reactor response. In the Polimi 
and TUDelft models, Fiorina et al. \cite{fiorina_modelling_2014}
stated that after around $t=15$ s, the ``flow pattern changed in the core and
the recirculation zones started to disappear''. The pocket of hot salt leaves
the core and causes a sudden drop in the average core temperature. In Moltres,
the wider peak in the average core temperature
indicates that there was a more gradual change in the flow pattern.

Figure \ref{fig:lofflowtemp} shows the flow patterns and temperature
distribution in the core at $t=300$ s in all three models. Figures
\ref{fig:lofheat}, \ref{fig:loftemp}, and \ref{fig:lofflowtemp} combined
highlight the difference between laminar flow in the Moltres model and
buoyancy-driven flow in the other models, and its impact on the reactor
response. They show that low-speed laminar flow is a poor substitute for
buoyancy-driven flow in the context of the MSFR. It is particularly evident
in the transition from high-speed turbulent flow to low-speed viscous
flow as Moltres mispredicted the intermediate stages. As for $\mu_t$, the need
to fine-tune this parameter confirms that the uniform, function-based $\mu_t$
approach is inappropriate for safety analysis in a loss of flow accident.

The results from this transient inform our goals for Moltres: 1)
implementing a proper turbulence model, and
2) developing a new heat exchanger feature that is compatible with the
buoyancy-driven flow capabilities already present in Moltres.

\begin{figure}[htbp!]
    \centering
    \includegraphics[width=.85\textwidth]{lof-heat}
    \caption{Power output during
    an unprotected loss of flow transient in the Moltres, Polimi, and
    TUDelft models \cite{fiorina_modelling_2014}.}
    \label{fig:lofheat}
    \includegraphics[width=.85\textwidth]{lof-temp}
    \caption{Average core temperature increase during
    an unprotected loss of flow transient in the Moltres, Polimi, and
    TUDelft models \cite{fiorina_modelling_2014}.}
    \label{fig:loftemp}
\end{figure}

\clearpage

\begin{figure}[htbp!]
    \centering
    \includegraphics[width=.37\textwidth]{lof-flow-temp}
    \includegraphics[width=.62\textwidth]{fiorina-lof}
    \caption{Temperature and velocity fields in the core at $t=300$ s during
    a loss of flow transient in the Moltres ($\mu_c = \frac{1}{2} \mu_{t,0}$),
    Polimi, and TUDelft models.}
    \label{fig:lofflowtemp}
\end{figure}
