Molten salt reactors is a class of advanced nuclear reactors that promise
numerous improvements over the current fleet of largely light-water reactors.
As the world continues its transition towards low-carbon electricity
generation to combat climate change, molten salt reactors is a potential
option in the near-term future for replacing fossil fuel and ageing nuclear
power plants. At the current state of development, molten salt reactors still
require extensive research to become viable. This thesis presents the latest
developments in Moltres, a simulation tool for molten salt reactors. These
new developments are: support for coupling the incompressible Navier-Stokes
and the delayed neutron precursor looping systems, and a model for
simulating decay heat from fission products at steady state and during
transients. This work demonstrates these capabilities through multiphysics
simulations of the Molten Salt Fast
Reactor concept. This work first verifies the six-group neutron diffusion
results from Moltres against continuous-energy Monte Carlo neutron transport
results from Serpent 2. The multiplication factors $k_{\text{eff}}$, delayed
neutron fractions $\beta$, temperature reactivity coefficient $\alpha_T$, and
the six-group neutron energy spectra from Moltres agreed with the high
fidelity simulation results from Serpent 2. The $k_{\text{eff}}$ values have
small discrepancies on the order of 100 pcm, which is smaller than the
$-256.7$ pcm discrepancy reported in literature with the same six-group
neutron diffusion approach. The decay heat model shows an expected flattening
of the temperature distribution due to the movement and dispersion of the
decay heat precursors throughout the primary coolant loop. This work also
demonstrates and verifies steady state and
transient multiphysics simulations of the Molten Salt Fast Reactor. The
transient scenarios under study were unprotected instances of reactivity
insertion, loss of heat sink, loss of flow, and pump overspeed. The steady
state and transient results were verified against results from another paper
that presented results for the same cases. The steady-state temperature and
velocity distributions, and the peak neutron flux showed good agreement with
the literature results. Minor differences in the delayed neutron precursor
distribution and the in-core delayed neutron fraction were explainable with
the differences in the handling of turbulence in the models. In three of the
transient results (reactivity insertion, loss of heat sink, and pump
overspeed), Moltres reproduced the expected magnitude and pattern of the
reactor response to these transient initiators.
The loss of flow results showed greater discrepancies that are attributed
to differences in the fluid dynamics modeling in Moltres and the other models.
Through the verification studies, this work has also identified avenues
for further Moltres software development.