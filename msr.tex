\glspl{MSR} are one of six advanced reactor designs shortlisted by
the \gls{GIF} in 2001 for promising significant advances in safety,
sustainability, efficiency, and cost over existing designs in operation
today. This has attracted significant attention and resources towards
\gls{MSR} research, most noticeable by the number of start-up companies that
have emerged in recent years touting various \gls{MSR} designs. This chapter
provides a brief history of \glspl{MSR}, followed by the distinctive features
that earned the concept the label of being a Generation IV reactor. Lastly,
we present the reference specifications of the \gls{MSFR} concept studied in
this work.

\section{History}

The first \gls{MSR}, named the \gls{ARE}, dates back to the 1940s,
as part of the US Aircraft Nuclear Propulsion program; the molten salt
concept was considered due to the stability of molten salts at high
temperatures and neutron radiation. The 2.5
MW$_{\text{th}}$ reactor was built at \gls{ORNL}, where it achieved
criticality on November 1954 and generated 100 MWh over nine days. The fuel
consisted of enriched uranium in a molten salt mixture of NaF, ZrF$_4$, and
UF$_4$, and moderated by blocks of beryllium oxide. The project ultimately
never came to fruition as the development of intercontinental ballistic
missiles effectively eliminated the need for long-range nuclear-powered
bomber aircraft.

However, the successful demonstration of the \gls{ARE} spurred further
research into adapting \glspl{MSR} for civilian power generation. One of the
key findings from the
research was that better economy could be achieved from breeding $^{233}$U
from $^{232}$Th in thermal-spectrum reactors than $^{239}$Pu from $^{238}$U.
Ultimately, these efforts culminated in the \gls{MSRE}, a graphite-moderated
thermal \gls{MSR}.
Although no breeding was attempted with the \gls{MSRE}, scientists at
\gls{ORNL} obtained a wealth of experimental data and new insights from the
study of this reactor. The \gls{MSRE} had a graphite-moderated design with
LiF-BeF$_2$-ZrF$_4$-UF$_4$ fuel salt mixture, initially rated at 10
MW$_{\text{th}}$ but later restricted to 8 MW$_{\text{th}}$ due to a
miscalculation of heat transfer capabilities. 

Design of the \gls{MSRE} commenced in the summer of 1960, with construction
starting in early 1962. The reactor achieved zero-power criticality in June
1965, and 30 days of continuous operation at full power in December 1966.
The reactor operated at full power for the most of the following 15 months,
during which the researchers carried out various experiments. Soon after
shutdown, the $^{235}$U fuel was replaced with $^{233}$U and in January
1969, the \gls{MSRE} became the first reactor to run on $^{233}$U fuel.

Although the \gls{MSRE} was a huge success, \gls{ORNL} failed to secure new
funding for the construction and operation of the \gls{MSBR}. The \gls{MSR}
development program lost out to the competing \gls{LMFBR} program which had
a head start and garnered wider political and technical support.
Nevertheless, from a technical perspective, two independent technology
evaluation and design studies of the \gls{MSR} had ``reported favorably on
the promise of the system".

\section{Features}

As mentioned in the introduction section, the most significant difference
between \glspl{MSR} and other reactor
concepts is the liquid fuel in \glspl{MSR}; fissile and/or fertile material
is dissolved in
high temperature, commonly eutectic mixtures of molten salts. Most \gls{MSR}
designs are circulating-fuel reactors. The primary coolant loop containing
the fuel salt transfers heat through a heat exchanger to the clean,
secondary/intermediate loop.

The flexibility of \glspl{MSR} is best illustrated by the various designs
under development today. Graphite-moderated thermal-spectrum \glspl{MSR} are
typically straight-forward \gls{LEU} burners, or $^{232}$Th/$^{233}$U
breeders, while epithermal- and fast-spectrum \glspl{MSR} have the additional
options of operating as \gls{TRU} fuel burners or $^{238}$U/$^{239}$Pu
breeders. Breeder designs can be further categorized into one- or two-fluid
designs; two-fluid designs feature blanket molten salt mixtures that contain
higher proportions of fertile material than the fuel salt mixture.

