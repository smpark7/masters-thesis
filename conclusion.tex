Unique phenomena in \glspl{MSR} necessitate the development of new reactor
safety analysis software. This thesis presents the latest developments in
Moltres, namely coupling its existing neutron diffusion module to the
incompressible Navier-Stokes module in MOOSE and developing a decay heat model
for short-term transients. This work demonstrated and verified some of
Moltres' current capabilities through a static-model
neutronics study and a coupled neutronics/thermal-hydraulics safety analysis
of the \gls{MSFR} concept.

The neutronics study showed good agreement between Moltres and Serpent 2. With
the relevant group constant data from Serpent 2,
Moltres could accurately replicate the $k_{eff}$, $\beta$,
$\alpha_T$, and multi-group neutron flux results from Serpent 2. The
$k_{eff}$ estimates from Moltres were approximately 100 pcm higher for all
measurements between 800 K and 1400 K. This discrepancy is notably smaller
than the discrepancies observed in the neutron diffusion and SP3 models
developed in OpenFOAM \cite{aufiero_extended_2013}. The $\beta$ and $\alpha_T$
values from Moltres had 1.46\% and 0.265\% discrepancies, respectively, to
Serpent 2's results. Lastly, the normalized six-group neutron fluxes from
Moltres all had less than 1\% discrepancy compared with Serpent 2.
The results of this study extend
the code-to-code verification of Moltres'
neutron diffusion model with the six-group, fast-spectrum \gls{MSFR} model.

Although Moltres currently lacks a proper turbulence model, the uniform
turbulent viscosity $\mu_t$ approximation was accurate enough to reproduce
most of the expected results for the \gls{MSFR}
steady-state and transient analyses. The steady-state temperature and velocity
distributions showed many similiarities in their shapes and magnitudes to the
Polimi and TUDelft model results \cite{fiorina_modelling_2014}. The uniform
$\mu_t$ approximation was responsible for the minor differences in the flow at
the top of the core and the loss of delayed neutrons to out-of-core emissions.
The results with the decay heat model showed a slight flattening of the
temperature distribution in the core that is in line with our expectations
given the diffusion and advection of the decay heat precursors.

The unprotected reactivity insertion and loss of heat sink results showed the
same trends Fiorina et al. \cite{fiorina_modelling_2014} observed in their
Polimi and TUDelft models. The
small difference in the temperature reactivity coefficient accounted for the
small difference in the magnitude of the peaks in power output and average
core temperature increase. The differences between the pointwise heat
exchanger in Moltres and the volumetric heat exchanger in the other two
models required minor adjustments in the relationship between flow rate
$\dot{V}$ and the heat transfer coefficient $h$ from the original
Dittus-Boelter correlation for the pump-initiated transients. Assuming a
directly proportional relationship between $\mu_t$ and $\dot{V}$ yielded
results in good agreement with the other two models for the pump overspeed
transient. However, Moltres performed poorly in the loss of flow transient as
this work could not incorporate buoyancy-driven flow and its associated
effects on $\mu_t$.

\section{Future Work}

Further research and development on Moltres should aim to rectify the issues
mentioned in this thesis. This work has highlighted three main avenues for
improvement. Firstly, proper 2D/3D heat exchanger implementation
would allow us to move away from the 1D outer loop system and towards a full
2D/3D closed loop. If this were implemented, users would be able to use
the Boussinesq approximation for buoyancy-driven flow capability in Moltres.
Buoyancy-driven flow is a critical component in loss of forced flow scenarios
which are important to reactor safety analyses.

Secondly, Moltres would benefit from a proper turbulence model such as
$k$-$\epsilon$ \cite{jones_prediction_1972} or $k$-$\omega$
\cite{wilcox_turbulence_2006}. Turbulence effects are
significant in \gls{MSR} designs with fast flow and they inform optimization
studies for improving flow patterns and eliminating local hotspots in eddies.
The current work, particularly the loss of flow accident scenario, shows
that simplifying assumptions for turbulence lead to erroneous results under
flow conditions that deviate significantly from steady state.

Lastly, a compressible Navier-Stokes model would be essential for modeling
effects such as variable temperature-dependent density
changes following a large reactivity insertion and finite wave propagation
speeds in the salt. The presence of bubbles in the core from the gas sparging
system increases fuel compressibility and enhances compressibility effects
\cite{cervi_development_2019}. 
