\section{Background and Motivation}

Greenhouse gas emission from human activities is the main cause of climate
change, which has dire
consequences on human health and safety due to extreme weather events and the
overall impact on food production \cite{mcmichael_global_2004}.
Electricity generation from burning fossil fuels represents the
greatest source of CO$_2$ emissions (38\% in 2018 \cite{iea_global_2019});
replacing it with low-carbon
alternatives would curb a significant fraction of emissions. Nuclear power is
a viable low-carbon replacement for burning fossil fuels and it provides
consistent base-load power independent of weather and geographical location
\cite{petti_future_2018}. Furthermore, employing a diverse mix of nuclear
power and renewable sources ensures energy security and reliability in our
transition towards a low-carbon future \cite{petti_future_2018}.

The world would have to ramp up the current rate of reactor deployments to
displace a portion of the presently large share of energy production from
fossil
fuel power plants. However, several obstacles stand in the way of mass reactor
deployments. These obstacles include perceived safety risks, sustainability
concerns, nuclear proliferation
risks, and the ability to compete economically with other sources of energy
\cite{massachusetts_institute_of_technology_future_2003}. A potential solution
to the aforementioned issues is the \gls{MSR} concept, one of six advanced
reactor designs selected by the Generation IV International Forum
\cite{gif_technology_2002} for continued research and development.

The primary coolants in MSRs consist of molten salt mixtures
with fissile and/or fertile material dissolved directly in the coolants.
MSRs possess an inherently robust safety feature in the strongly negative fuel
temperature coefficient of reactivity. Some designs can also incorporate the
thorium fuel cycle for improved sustainability arising from the use of
abundant natural thorium resources and reduced transuranic waste. The
latter also reduces economical costs
associated with long-term nuclear waste storage. In addition, the ability to
operate at near atmospheric pressures eliminates the need for a thick pressure
vessel and drives down construction costs, while online fuel reprocessing
reduces reactor downtime during reactor operation.

However, the liquid fuel form also brings about novel computational
challenges in simulating the transient behavior of \glspl{MSR}; the
interactions between neutronics and thermal-hydraulics are stronger due
to greater fuel material expansion. Furthermore, fissile material and
\glspl{DNP} in \glspl{MSR} can flow freely within the primary coolant
loop as opposed to being held in place in a solid fuel matrix. Therefore,
the choice of coupling methods for each set of physics requires careful
consideration. 

Most reactor analysis applications are usually reactor-specific by
design such as TRACE \cite{nrc_trace_nodate} for \glspl{LWR}, and
SAS4A/SASSYS-1 \cite{fanning_sas4a/sassys-1_2017} for
liquid metal cooled reactors. Thus, these applications would disregard
\gls{MSR}-specific phenomena and are inappropriate for \gls{MSR}
analysis without modifications to the source code. Some research efforts
do focus on adapting these applications for \gls{MSR} analysis. Examples
include the coupling of modified versions of TRACE and PARCS
\cite{pettersen_coupled_2016}, and the development of VERA-MSR from the
integrated \gls{LWR} simulation tool VERA \cite{graham_development_2019}.
Others developed their \gls{MSR} simulation tools from general
multiphysics or \gls{CFD} applications such as COMSOL
\cite{fiorina_modelling_2014} and OpenFOAM \cite{aufiero_development_2014}.

Similarly, Moltres \cite{lindsay_introduction_2018} is an open-source MSR
simulation tool built in the \gls{MOOSE} \cite{gaston_physics-based_2015}
parallel finite element framework. Lindsay et al.
\cite{lindsay_introduction_2018} first presented the tool in 2017 and
demonstrated its capabilities by simulating 2-D and 3-D models of the
\gls{MSRE}. The results showed good qualitative
agreement with the original design calculations by \gls{MSRE} researchers at
\gls{ORNL}. This thesis presents some of the new developments in Moltres
allowing for more complex and accurate \gls{MSR} simulations.

\section{Objectives}

This thesis demonstrates latest capabilities of Moltres
\cite{lindsay_introduction_2018}.
In particular, this thesis presents two more recent
developments in Moltres, namely fully integrating \gls{MOOSE}'s incompressible
Navier-Stokes module into Moltres, and introducing a
decay heat model.
The main objective of this thesis is to verify Moltres'
latest capabilities in modeling multiphysics, steady-state, and transient
behavior of fast-spectrum \glspl{MSR} through the study of the \gls{MSFR}
concept. Code-to-code verification is an important exercise in software
development for ensuring that the application produces accurate and reliable
results. This thesis covers the \gls{MSFR} concept mainly because it has been
studied extensively with readily available data in the literature to verify
against. The \gls{MSFR} design also features interesting flow
patterns that greatly affect the steady-state and transient behavior. This
present work will first present a verification of Moltres' \gls{MSFR}
diffusion neutronics against the Monte Carlo neutron transport software
Serpent 2, followed by a verification of
the coupled neutronics/thermal-hydraulics steady-state and accident transient
results against two sets of results published by
Fiorina et al. \cite{fiorina_modelling_2014}. The two sets of results arose
from a collaborative benchmarking exercise by researchers at Politecnico di
Milano and Technical University of Delft with two separate \gls{MSR}
simulation tools. Section \ref{sec:litrev} discusses these tools
in greater detail. The
secondary objective is to identify areas of improvement in Moltres for future
development.

\section{Thesis Outline}

The outline of this thesis is as follows. Chapter 2 discusses the history and
features of \glspl{MSR}, and a literature review of existing \gls{MSR}
simulation tools. The chapter also covers the \gls{MSFR} concept in greater
detail. Chapter 3 details the software and the general modeling
approach for generating the results in this thesis. Chapter 4 provides a
neutronics assessment by comparing key neutronics parameters from Moltres'
eigenvalue calculations to Serpent's Monte Carlo calculations. Chapter 5
presents steady-state results of coupled neutronics/thermal-hydraulics
\gls{MSFR} simulations in Moltres. Chapter 6 presents accident transient
simulation results for unprotected reactivity insertions, unprotected loss of
heat sink, unprotected loss of flow, and unprotected pump overspeed. Lastly,
Chapter 7 summarizes the key findings in this thesis
and posits some potential avenues for future work.
