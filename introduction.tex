\section{Background and Motivation}

Since the Industrial Revolution in the 1800s, rapidly rising greenhouse gas
emissions have contributed significantly to global warming, the consequences
of which have been acutely felt in recent years through increasingly frequent
and record breaking extreme weather events. Urgent measures are necessary to
limit CO$_2$ emissions and the negative impacts of global warming.

Since electricity generation from burning fossil fuels represents the
largest source of CO$_2$ emissions (38\% in 2018 \cite{iea_global_2019}),
replacing them with low-carbon
alternatives would effectively curb a large fraction of these emissions. Out
of all low-carbon power sources, nuclear power in general is best suited to
replace fossil fuel burning; it provides consistent base-load
power independent of weather and local geographical conditions. Solar and wind
power are dependent on favorable weather conditions while hydropower requires
relatively more area and is limited to geologically appropriate locations.
This argument provides a strong case for nuclear power being the most
appropriate and realistic alternative to burning fossil fuels for our current
and future electricity needs.

Large scale reactor deployments would be necessary to displace the
presently large share of energy production from fossil fuel power plants;
however, several obstacles stand in the way of mass reactor deployments. These
include perceived safety risks, sustainability concerns, nuclear proliferation
risks, and the ability to compete economically with other sources of energy.
The \gls{MSR} concept, one of six advanced reactor designs
selected by the \gls{GIF} \cite{gif_technology_2002} for continued research
and development, is a potential solution to the aforementioned issues.

The primary coolant in MSRs is a molten salt mixture,
with fissile and/or fertile material directly dissolved in the coolant.
MSRs possess an inherently robust safety feature in the
strongly negative fuel temperature reactivity coefficient from Doppler
broadening and thermal fuel expansion that greatly reduces
the risk of a reactor power excursion. Many designs can also incorporate the
thorium fuel cycle for improved sustainability arising from the use of
abundant natural thorium resources and reduced transuranic waste. The
resultant reduced radiotoxicity from transuranic waste also reduces costs
associated with long-term nuclear waste storage. In addition, the ability to
operate at near atmospheric pressures eliminates the need for a thick pressure
vessel and drives down construction costs, while online fuel reprocessing
reduces reactor downtime during reactor operation.

However, the liquid fuel form also brought about novel computational
challenges in simulating the transient behavior of \glspl{MSR}; the
neutronics and thermal-hydraulics are more tightly coupled due to the strong
temperature reactivity coefficient, and the additional advection coupling
terms. Furthermore, we have to account for the movement of \glspl{DNP} as
they are now generated directly within the primary coolant loop. Therefore,
the choice of coupling methods for each set of physics requires careful
consideration. Existing reactor system-level codes and modeling approaches
for conventional \gls{LWR} analysis contain simplifying assumptions in
multiphysics coupling and other areas that render the techniques unsuitable
for simulating \glspl{MSR}. Thus, making minor modifications to these codes
without changing the underlying algorithms is not the best approach for
\gls{MSR} safety analysis.

In the past two decades, researchers have developed several
new tools for simulating steady-state and transient behavior in \glspl{MSR}.
Many of the earlier efforts featured simplifications in simulating
thermal-hydraulics by solving 1-D Navier-Stokes equations or using
predetermined uniform velocity fields \cite{krepel_dyn3d-msr_2007}
\cite{kophazi_development_2009}. In more recent years, there has been
significant progress towards fully coupled, spatial codes that feature
2-D axisymmetric or full 3-D models. In 2011, Cammi et al.
\cite{cammi_multi-physics_2011} performed a ``\gls{MPM}'' analysis
of a simplified 2-D axisymmetric model of a single \gls{MSBR} fuel channel
using the commercial finite element analysis software COMSOL Multiphysics. The
physics were implemented through the two-group neutron diffusion
equations, and the \gls{RANS} standard $k-\epsilon$ turbulence model, for the
neutronics and thermal-hydraulics respectively. The
authors emphasized the need for proper full coupling of the multiphysics and
presented both steady-state and transient results in various
scenarios such as reactivity insertions, changes in pumping rate, and the
presence of periodic perturbations. This approach was featured again in a
later paper by Fiorina et al. \cite{fiorina_modelling_2014} in 2014 for a 2-D
axisymmetric model of the
\gls{MSFR}. The authors presented results from the Politecnico di Milano
COMSOL-based approach, and another approach by researchers from Delft
University of Technology, where they coupled their in-house neutronics and
thermal-hydraulics codes, DALTON and HEAT respectively. With multigroup
neutron diffusion and \gls{RANS} formulations on ultra fine meshes, both
models showed good agreement in the steady-state neutron flux, temperature,
and \gls{DNP} distributions, and in the power responses following various
accident transient initiations. Aufiero et al. \cite{aufiero_development_2014}
concurrently developed
a full-core 3-D model of the \gls{MSFR} on OpenFOAM, albeit with one-group
neutron diffusion to reduce computational load. With the 3-D model, the
authors could simulate the asymmetric reactor response to the failure of a
single pump in the sixteen-pump \gls{MSFR} configuration. The authors also
provided some quantitative data supporting the use of implicit coupling over
explicit coupling to obtain accurate solutions of the transient cases.
Recognizing the huge computational burden required for full 3-D simulations, 
later authors came up with innovative ways to alleviate this issue such as
selective geometrical reduced order modeling for various components of a
reactor based on the importance of the physical phenomena being simulated
\cite{zanetti_geometric_2015}, or using a novel, efficient method for
neutronics calculations \cite{laureau_transient_2017}.

The preceding discussion highlights the challenges faced in MSR simulation,
and the need for a highly efficient simulation tool that incorporates
implicitly coupled multiphysics with good computational scalability over
multiple processing units. This paper presents the open source MSR simulation
tool, Moltres, as a strong contender to overcoming these challenges for
simulating the MSFR. Moltres is an application code
built in the Multiphysics Object-Oriented Simulation Environment (MOOSE)
parallel finite element framework. Similar to COMSOL and OpenFOAM, it solves
the deterministic multigroup neutron diffusion and thermal-hydraulics
\glspl{PDE} simultaneously on the same mesh. It supports up to 3-D meshes and
scales well for a large number of processors.

\section{Objectives}

Moltres employs fully implicit coupling between the neutronics and
thermal-hydraulics governing equations, to fully account for the tightly
coupled physics expected in MSRs due to the movement of fuel in the salt. We
implemented flow using the Navier-Stokes equations and zeroth-order
approximation of eddy viscosity for a more accurate representation of the flow
profile compared to imposing a predetermined velocity field.

The main objective of this thesis is to demonstrate Moltres' capabilities in
modeling multiphysics, steady-state and transient behavior of fast-spectrum
\glspl{MSR} through the study of the \gls{MSFR} concept. This is achieved by
first verifying Moltres' neutronics results against Serpent in the context of
the \gls{MSFR}, and then comparing the coupled neutronics/thermal-hydraulics
steady-state and transient accident results of the \gls{MSFR} concept. The
multiphysics results are verified against the results by Fiorina et al.
\cite{fiorina_modelling_2014} and Aufiero et al.
\cite{aufiero_development_2014}.

\section{Thesis Outline}

The outline of this thesis is as follows: Chapter 2 discusses the history and
features of \glspl{MSR}. The \gls{MSFR} concept is covered in greater detail.
Chapter 3 details the simulation codes and the general modeling approach for
generating the results in this thesis. Chapter 4 provides a neutronics
assessment by comparing key neutronics parameters from Moltres' eigenvalue
calculations to Serpent's Monte Carlo calculations. Chapter 5 presents
steady-state results of coupled neutronics/thermal-hydraulics \gls{MSFR}
simulations in Moltres. Chapters 6 to 9 present transient accident simulation
results for unprotected reactivity insertions, unprotected loss of flow,
unprotected pump overspeed, and unprotected loss of heat sink, respectively.
Lastly, Chapter 10 summarizes the key findings of this thesis and posits some
potential avenues for future work.
