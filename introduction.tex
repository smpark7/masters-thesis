\section{Background and Motivation}

Rapidly rising greenhouse gas emissions since the Industrial Revolution in the
1800s have contributed significantly to global warming, the consequences of
which have been acutely felt in recent years through increasingly frequent and
record breaking extreme weather events. Urgent measures are necessary to limit
CO$_2$ emissions and the negative impacts of global warming.

Since electricity generation from burning fossil fuels represents the
largest source of CO$_2$ emissions (38\% in 2018 \cite{iea_global_2019}), replacing them with
low-carbon
alternatives would effectively curb a large fraction of these emissions. Out
of all low-carbon power sources, nuclear power in general is best suited to
replace fossil fuel burning; it provides consistent base load
power independent of weather and local geographical conditions. Solar and wind
power are dependent on favorable weather conditions, while hydropower requires
relatively more area and is limited
to geologically appropriate locations. This argument
provides a strong case for nuclear power being the most appropriate and
realistic alternative to burning fossil fuels for our current and future
electricity needs.

Large scale reactor deployments would be necessary to displace the
presently large share of energy production from fossil fuel power plants.
However, several obstacles stand in the way of mass reactor deployments. These
include perceived safety risks, sustainability concerns, nuclear proliferation
risks and the ability to compete economically with other sources of energy.
The Molten Salt Reactor (MSR) concept, one of six advanced reactor designs
selected by the Generation IV International Forum
(GIF)* for continued research and development, is a potential solution to the
aforementioned issues.

The primary coolant in MSRs is a molten salt mixture,
with fissile and/or fertile material directly dissolved in the coolant.
MSRs possess an inherently robust safety feature in the
strongly negative fuel temperature reactivity coefficient from Doppler
broadening and thermal fuel expansion that greatly reduces
the risk of a reactor power excursion. Many designs can also incorporate the
thorium fuel cycle for improved sustainability arising from the use of abundant
natural thorium resources and reduced transuranic waste. The resultant reduced
radiotoxicity from transuranic waste also reduces costs associated with
long-term nuclear waste storage.
In addition, the ability to operate at near atmospheric pressures eliminates
the need for a thick pressure vessel and drives down construction
costs, while online fuel reprocessing reduces reactor downtime during reactor
operation.

The development of MSR designs requires efficient and accurate simulation tools
for the complex neutronics and thermal-hydraulics behaviors expected in the
reactor core. Among other uses, these tools are essential in identifying and
characterizing potential accident scenarios to improve the safety of the design
and inform efforts towards creating MSR licensing frameworks.

Over the past decade, there has been a significant increase in the number of
simulation tools made available for transient MSR analyses. They generally fall
into two categories: loosely coupled, separate, stand-alone codes for
neutronics and thermal-hydraulic, and multiphysics software for concurrently
solving neutron transport and thermal-hydraulics. Fiorina et al.
\cite{fiorina_modelling_2014} presented simulation results from two
models developed independently at \textit{Politechnico di Milano} and
\textit{Technical University of Delft} in the context of the Molten Salt Fast
Reactor (MSFR). The MSFR is a circulating fuel reactor concept under active
development in the H2020 SAMOFAR project. The fuel consists of a molten salt
mixture of thorium and uranium fluorides dissolved in lithium fluoride (LiF).
The former employed the commercial
multiphysics modeling software COMSOL \cite{comsol_ab_comsol_2018}, while the
latter coupled two in-house codes DALTON and HEAT for neutron transport and
CFD calculations respectively. Both models showed generally good agreement with
each other with a simplified, 2-D axisymmetric model for several transient
scenarios.

Another paper by Aufiero et al. showed simulation results for a full 3-D model
of the MSFR using the multiphysics toolkit OpenFOAM. While running a 3-D model
is more computationally intensive, it allows for proper modeling of 3-D
turbulent effects, and the analysis of asymmetric transient scenarios such as
the single pump failure scenario in the MSFR (out of 16
pumps). The authors also demonstrated that there exists a clear discrepancy in
the results between implicit and explicit coupling schemes for the neutronics
and thermal-hydraulics. This is especially true for MSRs due to the strong
coupling arising from large temperature feedback coefficients. 

The preceding discussion highlights the challenges faced in MSR simulation, and
the need for a highly efficient simulation tool that incorporates implicitly
coupled multiphysics with good computational scalability over multiple
processing units. This paper presents the open source MSR simulation tool,
Moltres, as a strong contender to overcoming these challenges for simulating
the MSFR. Moltres is an application code
built in the Multiphysics Object-Oriented Simulation Environment (MOOSE)
parallel finite element framework. It solves the deterministic neutronics and
thermal-hydraulics governing equations with implicit
time-stepping on a single mesh for up to three dimensions. The MOOSE framework
is also highly parallelizable with its use of MPI for parallel computing.

\section{Objectives}

In this thesis, we demonstrate the coupled multiphysics modeling capabilities
of Moltres through the reference configuration of the MSFR design. Moltres
differs from many other codes used for simulating MSRs due to the fully
implicit coupling between the neutronics and thermal-hydraulics governing
equations, to fully account for the tight coupling expected in MSRs due to the
movement of fuel in the salt. The Reynolds-Averaged Navier-Stokes (RANS)-based
k-epsilon ($k-\epsilon$)
turbulence model was implemented for better simulating turbulence effects in
the high Reynolds number flows expected in MSFRs. The model simulates the mean
flow in turbulent flow with the introduction of two additional variables, the
turbulence kinetic energy ($k$) and the rate of dissipation of turbulence
energy ($\epsilon$). Moltres models
neutronics and thermal-hydraulics behaviors through the deterministic
multi-group neutron diffusion and convection-diffusion partial
differential equations respectively.

The objectives of this thesis are: 1) to demonstrate and verify the new
$k-\epsilon$ turbulence modeling feature in Moltres, 2) and to study
steady-state and transient cases with the reference MSFR model and identify
general characteristic behaviors in MSRs and fast spectrum reactors. The
results can be verified against common test cases for
turbulence models, and the MSFR simulation results by Rouch et al. and Aufiero
et al., both of which couple neutronics with the $k-\epsilon$ turbulence model.

** insert thesis outline here **