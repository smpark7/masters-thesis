Rapidly rising greenhouse gas emissions since the Industrial Revolution in the
1800s have contributed significantly to global warming, the consequences of
which have been acutely felt in recent years through increasingly frequent and
record breaking extreme weather events. Urgent measures are necessary to limit
CO$_2$ emissions and the negative impacts of global warming.

Since electricity generation from burning fossil fuels represents the
largest source of CO$_2$ emissions (38\% in 2018 \cite{noauthor_global_2018}), replacing them with
low-carbon
alternatives would effectively curb a large fraction of these emissions. Out
of all low-carbon power sources, nuclear power in general is best suited to
replace fossil fuel burning; it provides consistent base load
power independent of weather and local geographical conditions. Solar and wind
power are dependent on favorable weather conditions, while hydropower requires
relatively more area and is limited
to geologically appropriate locations. This argument
provides a strong case for nuclear power being the most appropriate and
realistic alternative to burning fossil fuels for our current and future
electricity needs.

Large scale reactor deployments would be necessary to displace the
presently large share of energy production from fossil fuel power plants.
However, several obstacles stand in the way of mass reactor deployments. These
include perceived safety risks, sustainability concerns, nuclear proliferation
risks and the ability to compete economically with other sources of energy.
The Molten Salt Reactor (MSR) concept, one of six advanced reactor designs
selected by the Generation IV International Forum
(GIF)* for continued research and development, is a potential solution to the
aforementioned issues.

The primary coolant in MSRs is a molten salt mixture,
with fissile and/or fertile material directly dissolved in the coolant.
MSRs possess an inherently robust safety feature in the
strongly negative fuel temperature reactivity coefficient from Doppler
broadening and thermal fuel expansion that greatly reduces
the risk of a reactor power excursion. Many designs can also incorporate the
thorium fuel cycle for improved sustainability arising from the use of abundant
natural thorium resources and reduced transuranic waste. The resultant reduced
radiotoxicity from transuranic waste also reduces costs associated with
long-term nuclear waste storage.
In addition, the ability to operate at near atmospheric pressures eliminates
the need for a thick pressure vessel and drives down construction
costs, while online fuel reprocessing reduces reactor downtime during reactor
operation.




