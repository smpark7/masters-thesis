The main simulation tool used in this work for studying the \gls{MSFR} is
Moltres. While we used Moltres for the steady state and transient analysis, it
requires a dedicated neutron transport solver such as Serpent 2 to generate
neutron energy group constant data for the deterministic, multi-group
neutronics calculations in Moltres. This chapter provides brief descriptions
of the Serpent 2 code and Moltres.

\section{Serpent 2}

Serpent 2 is a continuous-energy Monte Carlo reactor physics code under
active development at the VTT Technical Research Centre of Finland. It was
born in 2004 out of the need for better general-purpose neutron transport
codes for group constant generation in lattice geometries, and has since grown
to support more general capabilities, for an active user base of more than 500
people worldwide. Serpent is highly parallelizable due to its support for
both MPI and OpenMP parallel programming APIs, and highly validated and
verified against experimental data and well-established codes.

In Serpent, each neutron is tracked through a combination of ray-tracing-based
surface tracking and rejection sampling-based delta-tracking. Users may define
the number of neutrons and the number of active and inactive cycles for each
simulation. Inactive cycles are required for fission source distribution
convergence, before interactions are tallied in the active cycles.
Interaction types and locations are
determined stochastically based on neutron interaction data from established
nuclear data libraries (e.g. ENDF, JEFF). These nuclear data libraries provide
continuous-energy cross section data at discrete temperatures. For other
temperatures, Serpent has a built-in Doppler-broadening preprocessor that
extrapolates the relevant cross section data from a lower temperature. 

In the context of this project, we used Serpent with the JEFF-3.1.2 nuclear
data library* to generate group constants
for Moltres. The reactor geometry is based on the axisymmetric \gls{MSFR}
neutronics benchmark geometry published in several previous papers involving
the \gls{MSFR}. The relevant group constant data are collapsed into six
neutron energy groups, provided in table *, and calculated for temperature
values of 900 K to 1200 K at 50 K intervals.

The group constants relevant for neutronics calculations in Moltres are the
macroscopic fission, absorption, and scattering neutron cross sections,
neutron diffusion coefficients, average fission energies, average neutron
yields, inverse neutron speeds, flux spectra, \gls{DNP} decay constants, and
effective delayed neutron fractions. These group constants are extracted from
the Serpent output files using a Python script available from the Github
repository that holds the Moltres source code. The script rewrites the group
constants into a Moltres-compatible format.

\section{Moltres}

Moltres is a \gls{MOOSE}-based application for coupled
neutronics/thermal-hydraulics simulations of \glspl{MSR}. \gls{MOOSE} is a
highly parallelizable,
finite-element framework developed at \gls{INL} for simplifying the process of
creating fully-coupled, non-linear, multiphysics solvers. The framework
provides a user-friendly interface for this task through object-oriented
programming in C++. All aspects of a typical multiphysics problem, such as the
terms in the \glspl{PDE}, the initial and boundary conditions, the material
properties, etc., are represented in \gls{MOOSE} as C++ objects. This approach
is helpful for many researchers, a significant fraction of whom do not possess
high-level programming backgrounds, as they are unencumbered by the
technical details and complexicities involved in programming mesh handling
and \gls{PDE} solving in finite element analysis.

\gls{MOOSE} itself relies heavily on LibMesh and PETSc for
its mesh handling and \gls{PDE} solver functionalities. As a result,
\gls{MOOSE} supports adaptive meshing schemes and automatic variable scaling
amongst other features for improved accuracy and performance times. Full
coupling is maintained by the execution of Newton-based solves on the
weak formulations of the multiple \glspl{PDE} to minimize the residual values.
Fully-coupled solves are essential for accurately resolving systems with
strongly interacting physics. The \gls{MSR} concept is one such example, where
the neutronics and thermal-hydraulics are tightly coupled through the Doppler
effect and the temperature dependence of liquid fuel salt density.

\gls{MOOSE}, and Moltres by extension, are capable of up to 3D geometry
modelling. They support a wide range of input mesh file formats listed in a
\gls{MOOSE} webpage, including the commonly used Exodus II file format.
Specifically for the 2D case, axial symmetry is easily imposed by changing
one line of code in the input file, without any changes in the Cartesian
representations of the \glspl{PDE} and boundary conditions in their original
C++ implementations. This feature provides significant computational time
savings for 3D systems that exhibit high axial symmetry. Another important
feature for reducing computational time is the use of MPI for parallel
computing. All \gls{MOOSE}-based codes can be easily compiled and run on high
performance computing clusters.

Both \gls{MOOSE} and Moltres are open source codes licensed under Lesser GNU
Public License (LPGL) terms. In the case of Moltres, this was done in support
of open and transparent research practices as well as to leverage on the ease
of access for peer review and encourage continued development of the code.
Moltres is hosted on a Github repository of the same name. Github provides
version control and continuous integration testing, which ensure existing
physics capabilities are protected from potential bugs inadvertently
introduced when developing new physics.

\subsection{Neutronics}

Focusing our attention on Moltres, the neutron flux calculations are
performed using the standard formulations for the time-dependent multi-group
neutron diffusion equations and \gls{DNP} concentration equations as shown in
equations \ref{eq:neut} and \ref{eq:dnp}:
%
\begin{align}
	\frac{1}{v_g} \frac{\partial \phi_g}{\partial t} &= \nabla \cdot D_g
	\nabla \phi_g - \Sigma^r_g \phi_g
	\sum^G_{g \neq g'} \Sigma^s_{g' \rightarrow g} \phi_{g'} + \chi^p_g
	\sum^G_{g'=1} (1-\beta) \nu \Sigma^f_{g'} \phi_{g'} + \chi^d_g \sum^I_i
	\lambda_i C_i, \label{eq:neut} \\
	\frac{\partial C_i}{\partial t} &= \beta_i \sum^G_{g'=1} \nu \Sigma^f_{g'}
	\phi_{g'} - \lambda_i C_i - \boldsymbol{u} \cdot \nabla C_i + \nabla \cdot
	\frac{\mu_t}{\rho Sc_t} \nabla C_i, \label{eq:dnp} \\
	\intertext{where} v_g &= \text{average speed of neutrons in group }g 
	\nonumber \\
	\phi_g &= \text{flux of neutrons in group }g \nonumber \\
	t &= \text{time} \nonumber \\
	D_g &= \text{diffusion coefficient of neutrons in group }g \nonumber \\
	\Sigma^r_g &= \text{macroscopic cross-section for removal of neutrons from
	group }g \nonumber \\
	\Sigma^s_{g' \rightarrow g} &= \text{macroscopic cross-section of
	scattering from }g' \text{ to }g \nonumber \\
	\chi^p_g &= \text{prompt fission spectrum neutrons in group }g \nonumber
	\\
	G &= \text{total number of discrete neutron groups} \nonumber \\
	\nu &= \text{average number of neutrons produced per fission} \nonumber \\
	\Sigma^f_{g} &= \text{macroscopic fission cross-section for neutron in
	group }g \nonumber \\
	\chi^d_g &= \text{delayed fission spectrum neutrons in group }g \nonumber
	\\
	I &= \text{total number of delayed neutron precursor groups} \nonumber \\
	\beta &= \text{total delayed neutron fraction} \nonumber \\
	\beta_i &= \text{delayed neutron fraction of precursor group i} \nonumber
	\\
	\lambda_i &= \text{average decay constant of delayed neutron precursors in
	precursor group }i \nonumber \\
	C_i &= \text{concentration of delayed neutron precursors in precursor
	group }i \nonumber
\end{align}
%

The limitations of the multi-group neutron diffusion compared to
S$_{\text{N}}$ and Monte Carlo methods are well-documented but they provide
decent accuracy at lower computational costs. A comparison of the
multiplication factor and reactivity coefficients in the context of the
\gls{MSFR} is provided in the results section.

The \gls{DNP} concentration equation has an additional advection term to
account for the movement of \glspl{DNP} in the primary coolant loop. Moltres
also accounts for the decay of \glspl{DNP} outside the active core
region by simulating its flow in a separate 1D geometry. The length and
velocity defined in this geometry may differ according to reactor
specifications.

Moltres users may call for an arbitrary number of neutron energy groups as
long as they provide Moltres with the appropriate group constant data. The
number of precursor groups is also arbitrary, though usually pre-determined by
the choice of nuclear data library in the group constant generation step.
Moltres automatically interpolates the group constant data for required
temperatures using one of the many pre-defined interpolation methods available
in \gls{MOOSE}. Once again, users have the freedom to select their
interpolation method of choice. For this work, we have selected six neutron
energy groups according to the energy boundaries in table *, eight \gls{DNP}
groups as defined by the JEFF-3.1.2 library, and the spline interpolation
method.

For the decay heat model, previous studies on the MSFR * have shown that
using three decay heat precursor groups with different half-lives can model
decay heat in the MSFR for up to 300 seconds after shutdown with a relative
error of less than 2 \%. Thus, we implemented the decay heat model using
equations similar to the \gls{DNP} equations, as shown in eq
\ref{eq:decayheat}:
%
\begin{align}
	\frac{\partial D^k}{\partial t} &= f_k \sum^G_{g'=1} \epsilon_{g'}
	\Sigma^f_{g'} \phi_{g'} - \lambda^k D^k - \boldsymbol{u} \cdot \nabla D^k
	+ \nabla \cdot \frac{\mu_t}{\rho Sc_t} \nabla D^k, \label{eq:decayheat} \\
	\intertext{where}
	D^k &= \text{concentration of decay heat precursors in group }k \nonumber
	\\
	f_k &= \text{fraction of decay heat to total power at steady state}
	\nonumber \\
	\epsilon_g &= \text{average fission energy} \nonumber \\
	\lambda^k &= \text{average decay constant of decay heat precursors in
	group }k. \nonumber
\end{align}


\subsection{Thermal-Hydraulics}

The governing equation for temperature in the molten salt is given as:
%
\begin{align}
	\rho c_{p} \frac{\partial T}{\partial t} &= - \nabla \cdot \big( \rho
	c_{p} \overrightarrow{u} \cdot T - k \nabla T \big) + Q_s - Q_{hx},
	\label{eq3}
	\intertext{where}
	\rho &= \text{density of molten salt} \nonumber \\
	c_{p} &= \text{specific heat capacity of molten salt} \nonumber \\
	T &= \text{temperature of molten salt} \nonumber \\
	\overrightarrow{u} &= \text{velocity of molten salt} \nonumber \\
	k &= \text{thermal conductivity of molten salt} \nonumber \\
	Q_{hx} &= \text{heat sink},
\end{align}
	and the source term $Q_s$ is given as:
%
\begin{align}
Q_s = \sum^G_{g=1} \epsilon_g \Sigma_g^f \phi_g. \label{eq4}
\end{align}

In the \gls{MSFR}, salt flow in the core has a Reynolds number on the order of
10$^6$*. Thus the Navier-Stokes module in \gls{MOOSE} cannot sufficiently
capture the turbulent flow behavior. Furthermore, the \gls{MSFR} benchmark
geometry encourages the formation of turbulent recirculation regions in the
fuel salt near the blanket tank regions. Due to this present limitation, only
the flow in the active core region is simulated and it is assumed to be
vertical flow. This eliminates the presence of recirculation zones reported by
*, which will likely be a source of discrepancy between our results and
theirs. Another consequence is the absence of turbulent diffusion, resulting
in higher-than-expected temperatures along the central axis of the reactor
where the neutron flux and heat generation is highest. This is further
exacebated by the high Prandtl number of the molten salt; heat transfer is
dominated by convective flow. This means that there would be very low heat
transfer observed by diffusion due to the negligible contributions from
heat conduction.

For steady-state solves, there are two options: the inverse power
method (\texttt{InversePowerMethod}) and the nonlinear Newton method
(\texttt{NonlinearEigen}). 
